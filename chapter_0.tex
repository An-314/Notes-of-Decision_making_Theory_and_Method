\chapter{课程概要}

\section{课程简介}

决策过程。

\begin{example}
    博弈论中的囚徒困境。

    \href{https://zh.wikipedia.org/wiki/%E5%9B%9A%E5%BE%92%E5%9B%B0%E5%A2%83}{囚徒困境}反应个人理性和集体理性的矛盾。\href{https://zh.wikipedia.org/wiki/%E7%BA%B3%E4%BB%80%E5%9D%87%E8%A1%A1}{纳什均衡}是非合作博弈均衡。囚徒困境是混合策略纳什均衡:混合策略是给每一个纯策略分配一个概率,玩家的策略集是一个样本空间。

    例如:AB互相同时给出硬币的正反面,表格内(a,b)为在该情况下A、B的获利:
    \begin{table}[htbp]
        \centering
        \begin{tabular}{|c|c|c|}
            \hline
            A$\diagup $B & 正     & 反     \\
            \hline
            正            & +3,-3 & -2,+2 \\
            \hline
            反            & -2,+2 & +1,-1 \\
            \hline
        \end{tabular}
    \end{table}

    B同学用Nash均衡选择自己的策略:他决定出正面的概率为$p$。从而:

    A出正的时候B的获利:$3p - 2(1-p) $,A出反的时候B的获利:$-2p + (1-p) $。Nash均衡意味着二者相等,得到 $p=3/8$

    可以进一步地算出B的获利期望。
\end{example}

\section{课程内容}

\subsection{Markov决策过程(MDP)}

系统介绍复杂大系统决策方法的理论和应用。具体内容包括Markov决策过程,动态规划算法,实时数值迭代模拟方法,策略搜索方法,以及在大型突发事件应急管理过程中的动态资源分配、排队网络的应用方法。

重点介绍的分析内容及主要方法包括:

\begin{enumerate}[itemsep=0pt,parsep=0pt]
    \item 有限阶段问题
    \item 折扣成本问题
    \item 数值迭代方法
    \item 策略迭代方法
\end{enumerate}

\subsection{数据驱动的风险量化分析与建模}

基于历史灾害数据,通过学习统计分析、机器学习等多因素分析方法,实现对环境、社会等多种因素致灾风险的定量评估,并构建相应风险评估模型, 为决策制定提供数据基础。

本部分重点介绍的主要分析方法将包括:

\begin{enumerate}[itemsep=0pt,parsep=0pt]
    \item 混合线性模型(LME)
    \item 固定效应、随机效应、混合效应
    \item Logistic回归模型
    \item 广义可加模型(GAM)
    \item 强化机器学习模型
\end{enumerate}

\subsection{群决策与社会选择理论}

在个体决策的基础上, 本部分将进一步介绍多主体系统的决策理论和方法。 该部分将主要介绍多主体(Multi-Agent) 之间交互,以及多主体之间自身利益最大化, 以及相互协作等不同决策策略, 多主体决策理论是博弈论的主要分支。

本部分重点介绍的主要分析方法将包括:

\begin{enumerate}[itemsep=0pt,parsep=0pt]
    \item 群决策与社会选择理论
    \item 群决策概论
    \item 博弈论方法及算法
    \item 多目标群决策方法
    \item 冲突分析理论和方法
    \item 社会选择理论
\end{enumerate}

\subsection{数字化决策方法}

整合并清洗,整理,归类数据到基础数据仓库或者数据湖,集合先进模型、算法以及人工智能的数据分析软件工具,全面贯彻数字化概念,依靠数据分析进行决策的方法。

本部分重点介绍的主要分析方法将包括:

\begin{enumerate}[itemsep=0pt,parsep=0pt]
    \item 疫情预测模型
    \item 舆情分析模型
    \item 神经网络模型
    \item 深度学习模型
\end{enumerate}